\documentclass{article}
\usepackage[[catalan]{babel}
\usepackage[utf8]{inputenc}
\begin{document}
\title{Practical home multihoming with OpenWRT and Open Overlay Router}
\maketitle
\tableofcontents
\section{Definició de l’abast i contextualització}
\subsection{Context}
\subsubsection{Introducció}
Aquest és un projecte de Treball de Fi de Grau (TFG) del Grau en Enginyeria Informàtica cursat a la Facultat d’Informàtica de Barcelona (FIB) de la Universitat Politècnica de Catalunya (UPC), i desenvolupat en el marc d’un projecte de col·laboració entre el Departament d’Arquitectura de Computadors (DAC) i Cisco Systems, anomenat Open Overlay Router (OOR).
El projecte OOR es basa en l’arquitectura de xarxa Locator/ID Separation Protocol (LISP [1]) implementada per Cisco Systems, la qual vol solucionar el problema d’escalabilitat d’Internet causat per la creixent quantitat de dispositius que s’hi connecten i la manca d’adreces per identificar-los. Per fer-ho, LISP separa semànticament els dos tipus de dispositius que s’hi connecten: Els dispositius finals, que s’identifiquen mitjançant Endpoint Indentifiers (EIDs), i els dispositius que formen part de la capa d’enrutament, identificats per Routing Locators (RLOCs).
La intenció d’OOR és oferir una implementació de codi obert de LISP compatible amb Linux, Android i OpenWRT, per tal que tothom que vulgui pugui utilitzar aquest protocol.
OpenWRT, concretament, és una distribució de Linux per a dispositius incrustats (Majoritàriament routers i mòdems), que ofereix unes opcions de configuració molt més extenses i personalitzables que els firmwares preestablerts habituals [2]. Una d’aquestes opcions és, com a exemple rellevant per aquest projecte, el multihoming.
S’anomena multihoming a la capacitat d’un dispositiu de connectar-se a més d’una xarxa de computadors d’un o varis proveïdors d’Internet (ISP), sigui per augmentar la banda ampla disponible o la fiabilitat de la xarxa. Però per a poder-ho fer, el router en qüestió necessita un mínim de dos ports de Wide Area Network (WAN), i els routers domèstics habituals només en tenen un, limitant així la utilització d’aquesta funció a una gran part del mercat. Tot i això, mitjançant una personalització intensiva de la configuració d’OpenWRT, podem convertir un dels varis ports Local Area Network (LAN) d’un dispositiu domèstic en WAN mitjançant l’ús de Xarxes Locals Virtuals (VLANs). Malgrat això, aquesta configuració requereix amplis coneixements de xarxes de computadors i aconseguir-ho no està a l’abast de tothom.
L’objectiu d’aquest projecte és crear una distribució basada en OpenWRT per un model de router domèstic concret, que integri OOR i que permeti multihoming, configurant la mínima quantitat de paràmetres possible. D’aquesta manera, qualsevol usuari que vulgui gaudir de les millores d’aquesta arquitectura ho podrà fer de la manera més còmode i senzilla.

\subsubsection{Actors implicats}
Mitjançant la creació d’aquesta distribució es facilita enormement l’adopció d’aquest protocol a qualsevol àmbit, i això afecta al rol de vàries persones.
\paragraph{Desenvolupador principal}
Serà la única persona que desenvolupi el projecte, i s’encarregarà d’assolir totes les fites programades, documentar el projecte i finalment presentar-lo.
\paragraph{Equip d’OOR}
Quan el projecte estigui finalitzat, els dubtes i problemes derivats de la instal·lació i la posada en marxa d’OOR pels usuaris serà molt menor, ja que només caldrà indicar quin model de router han de comprar i proporcionar el binari instal·lador per el model concret. I també se’n beneficiaran en quan a base d’usuaris,  ja que la barrera de la dificultat en la seva preparació es veurà molt reduïda.
\paragraph{Cisco Systems}
Un altre beneficiat d’aquest projecte serà Cisco Systems, ja que permetrà que més routers utilitzin LISP, incrementant així el seu mercat i la seva competitivitat davant altres protocols similars.
\paragraph{Usuari final}
Degut a la fàcil instal·lació i la mínima configuració dels elements necessaris per utilitzar OOR, l’usuari final que vulgui fer-ne ús només haurà de seguir un petit manual en comptes d’haver d’invertir hores en preparar-ho, així que també se’n veuran beneficiats.

\subsection{Estat de l'art}
\subsubsection{LISP i OOR}
LISP es troba en un desenvolupament continu, optimitzant diferents paràmetres de l’arquitectura per a ser aplicats a resoldre varis problemes actuals [3], [4]. El seu objectiu és millorar tota l’estructura interna d’Internet, i per tant és una arquitectura molt ambiciosa i es troba molt avançada tecnològicament respecte a l’arquitectura que hi ha implementada ara mateix. 
El projecte OOR també es troba actualment en un estat molt actiu i avançat. Tot i que va començar sent una implementació de mobilitat per LISP [5], la seva aportació ha augmentat dràsticament, integrant en la seva implementació tecnologies noves com Open DayLight (ODL) [5] i . Ofereix funcionalitat completa a totes les plataformes mencionades anteriorment (Linux, Android i OpenWRT).
\subsubsection{OpenWrt}
La comunitat d’OpenWRT és una de les més actives en el sector de les distribucions enfocades per routers i mòdems. Per dur a terme el projecte, s’utilitzarà la seva última versió (15.05 Chaos Calmer [5]), i es buscarà un router compatible amb el sistema que estigui a la vanguardia de característiques per a que l’elecció no quedi ràpidament obsoleta.

\subsection{Abast del projecte}
\subsubsection{Objectius}
Com ja s’ha explicat, el projecte pretén facilitar el màxim possible la instal·lació i la posada en marxa de OOR. Per fer-ho, es seguiran els següents passos principals:
\begin{enumerate}
\item Es buscarà un model de router relativament nou on centrar la distribució, que sigui compatible amb la última versió d’OpenWRT, que no tingui mòdem intern (Ja que no compliria cap funció), i que permeti crear i gestionar VLANs.
\item Es cercarà la manera d’integrar tota la configuració de multihoming i d’OOR adequada dins d’un sol binari per facilitar al màxim la instal·lació a l’usuari final.
\item Es modificarà la interfície gràfica que proporciona OpenWRT perquè reflexi també els paràmetres d’OOR personalitzables.
\subsubsection{Possibles obstacles i solucions}
Durant el projecte poden sortir certes complicacions i dificultats que afectin el seu desenvolupament i modifiquin la seva planificació temporal.
\begin{enumerate}
\item La falta de disponibilitat del router escollit pot provocar un retard en el  desenvolupament. Solució: Començar a fer proves amb un altre router compatible.
\item Errors de hardware al router on s’hi treballarà. Solució: Aconseguir-ne més d’una unitat, si és possible, per si una falla.
\item Errors amb l’ordinador on es treballarà i possible pèrdua d’informació. Solució: El codi també s’emmagatzemarà en un gestor de versions.

\subsection{Metodologia}
\subsubsection{Mètode de treball}
S’utilitzarà la metodologia àgil Scrum, que ens permetrà gestionar el projecte de manera més amena i motivadora que amb un mètode més tradicional, ja que és adaptable a canvis i imprevistos i millora substancialment la productivitat. Tindrem reunions setmanals amb la resta de l’equip d’OOR.
\subsubsection{Eines de seguiment}
Quan es comenci a generar codi, aquest es gestionarà a través de Github, per així obtenir un control de versions i poder gestionar les modificacions i els canvis de la millor manera. A més, també permet treballar en local i en remot de manera fàcil i ràpida simplement tenint sincronitzat el repositori en tot moment a cada dispositiu des d’on es treballi.
També es seguirà un diagrama de Gantt per veure com avancen les tasques segons la planificació inicial.
\subsubsection{Validació}
Tota la feina es desenvoluparà en una sala amb gran part de l’equip d’OOR, per tant qualsevol dubte al respecte podrà ser resolt de manera més o menys immediata, i les reunions setmanals serviran per validar la feina feta durant la setmana i discutir quin és el següent pas a seguir.


\section{Planificació temporal}
\subsection{Descripció de les tasques}
\subsubsection{Gestió del projecte}
La facultat requereix que, quan s’elabora un Treball de Fi de Grau (TFG), també es dugui a terme una assignatura addicional,  anomenada Gestió de Projectes (GEP). El seu objectiu és ajudar a encaminar el projecte i començar a documentar-lo des del principi, per tal que la memòria sigui complerta i la seva defensa posterior es realitzi correctament. A continuació es detallen els set lliurables a entregar, juntament amb les hores estimades per realitzar cada tasca.
\begin{enumerate}
\item Definició de l’abast i contextualització (7.1 hores)
\item Planificació temporal (4.6 hores)
\item Gestió econòmica i sostenibilitat (6.1 hores)
\item Presentació preliminar (3.6 hores)
\item Plec de condicions (7.6 hores)
\item Document final (12.1 hores)
\item Presentació final (11.1 hores)
Sumant les hores indicades i les d’estudi de la matèria pertinents per cada lliurable, s’estima que la duració total serà de 75.2 hores.
Per dur-ho a terme s’utilitzarà un ordinador portàtil, Google Drive (Per editar els lliurables des de qualsevol altre ordinador si fos necessari), un mòbil amb càmera (Per editar el vídeo de la presentació preliminar), Dropbox (Per compartir el vídeo de la presentació preliminar), Libre Office (Per editar les rúbriques a entregar amb cada lliurable), Foxit Reader (Per visualitzar els PDFs amb les rúbriques), Atenea (On entregar els lliurables), i el Racó de la FIB (On entregar el document final).
\subsubsection{Cerca del model de router}
El primer pas d’aquest projecte és escollir un router on centrar el seu desenvolupament, ja que el sistema OpenWRT depèn totalment del hardware d’aquest. Es durà a terme analitzant la llista de dispositius compatibles amb el sistema [5], i haurà de complir amb els requisits que es detallen a continuació. El temps estimat a dedicar a la seva cerca és de 60 hores, i es necessitarà un ordinador proporcionat pel DAC, accés a Internet i Google Drive per llistar els models canditats abans de l’elecció.
\paragraph{Compatible amb la última versió d’OpenWrt}
Ja que el projecte està orientat a perdurar el màxim possible, el router haurà de ser compatible amb la versió més nova del sistema, per així no haver d’actualitzar la versió al cap de poc.
\paragraph{Disponible internacionalment}
El model escollit ha d’estar a la venta actual i internacionalment, ja que tothom ha de poder comprar-lo des d’arreu del món. Per tant, s’evitaran marques de producció només nacional o routers que ja no estiguin actualment en fabricació i distribució.
\paragraph{Compatible amb VLANs}
El projecte necessita que el hardware amb el que es treballarà pugui crear i gestionar xarxes virtuals, i així poder configurar el multihoming definit al document anterior.
\paragraph{Especificacions adequades}
Si el router és dels últims del mercat i les seves característiques a nivell de hardware són prou altes, no caldrà renovar-lo a curt ni mig plaç, ja que el que ofereixin els nous models estaran igualats als seus.
\subsubsection{Familiarització amb el sistema OpenWrt}
Al ser una distribució que no s’utilitza durant la carrera, s’hauran de dedicar vàries hores a familiaritzar-se amb el sistema i els seus dos mètodes d’interacció: UCI i LUCI. S’hi dedicaran aproximadament 60 hores en total, i serà necessari l’ordinador del DAC mencionat anteriorment, un mínim de tres targes de xarxa connectades a ell, un mínim de tres cables de xarxa d’un metre, el model de router escollit i accés a Internet.
\subsubsection{Familiarització amb LISP i OOR}
L’arquitectura LISP i el projecte OOR que l’utilitza són nous i tampoc es veuen a cap assignatura del grau, per tant caldrà també una familiarització amb ells i amb els seus fitxers de configuració (Tant a Linux com a OpenWRT). Es pereveu dedicar-hi unes 60 hores. Per a fer-ho, caldrà l’ordinador del DAC amb les targes de xarxa anteriors, cables de xarxa, el router escollit, vàries màquines virtuals i accés a Internet
\subsubsection{Configuració del multihoming}
El següent pas serà configurar un dels ports LAN del router, aïllat de la resta per una VLAN, com a port WAN. Això requerirà una estructura de xarxa OOR funcionant (Amb vàries màquines virtuals implicades i ben configurades) i una configuració del router OpenWrt  correcta, així que necessitarà que totes les tasques anteriors hagin acabat. S’hi dedicaran 70 hores. En quant a requeriments físics, caldrà l’ordinador del DAC amb les seves targes, cables de xarxa, dues línies WAN amb una adreça pública pròpia (S’usaran IPs de la UPC), el router i màquines virtuals.
\subsubsection{Automatització del multihoming}
La base del projecte és facilitar al màxim la configuració d’OOR i el multihoming a l’usuari final, per tant que aquest s’hagi de limitar a seguir el mínim de passos possible quan compri el router indicat. Per tant es buscarà la manera d’automatitzar tot el procés realitzat anteriorment, mitjançant scripts o amb la configuració prèvia d’arxius de OpenWrt OOR, i empaquetar-ho tot en una distribució independent. El temps estimat per a aquesta recerca i implementació serà de 70 hores, i es necessitarà també l’ordinador del DAC amb les seves targes de xarxa, els cables de xarxa, les dues línies WAN amb adreça pública pròpia, el router i màquines virtuals.
\subsubsection{Adaptació de la interficie d’OpenWrt}
Al haver afegit multihoming i OOR integrats directament al sistema, s’haurà de modificar la interficie de configuració que usarà l’usuari final. Aquesta haurà de comptar amb paràmetres per configurar els detalls mencionats anteriorment i facilitar la seva modificació. S’hi dedicaran 60 hores i es requerirà l’ordinador del DAC amb les seves targes de xarxa, cables de xarxa, dues línies WAN amb adreça pública pròpia, el router i màquines virtuals.
\subsubsection{Documentació i preparació de la defensa}
Finalment el projecte s’ha de documentar i defensar davant d’un tribunal de professors i doctors de la facultat. La redacció del document i la preparació de la defensa durarà unes 50 hores, i s’utilitzarà un ordinador portàtil o el proporcionat pel DAC, un editor de text compatible amb el llenguatge LaTeX, i un conversor de documents a PDF.

\subsection{Estimació del temps i seqüència de tasques}
\subsubsection{Temps estimat per tasca}
\begin{center}
	\begin{tabular}{| l | l }
		\hline
		Tasca & Hores estimades \\ \hline
		Gestió del projecte & 75.2 \\ \hline
		Cerca del model de router & 60 \\ \hline
		Familiarització amb OpenWrt & 60 \\ \hline
		Familiarització amb LISP i OOR & 60 \\ \hline
		Configuració del multihoming & 70 \\ \hline
		Automatització del multihoming & 70 \\ \hline
		Adaptació de la interficie d'OpenWrt & 60 \\ \hline
		Documentació i preparació de la defensa & 60 \\ \hline
		Total & 515 \\
		\hline
	\end{tabular}
\end{center}



\end{document}